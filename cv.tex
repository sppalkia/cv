%% Copyright 2006-2013 Xavier Danaux (xdanaux@gmail.com).
%
% This work may be distributed and/or modified under the
% conditions of the LaTeX Project Public License version 1.3c,
% available at http://www.latex-project.org/lppl/.

% possible options include font size ('10pt', '11pt' and '12pt'), paper size
% ('a4paper', 'letterpaper', 'a5paper', 'legalpaper', 'executivepaper' and
% 'landscape') and font family ('sans' and 'roman')
\documentclass[11pt,letterpaper,sans]{moderncv}
\usepackage{xspace}
\newcommand\eat[1]{}

%\newcommand{\ie}{{\it i.e.,}\xspace}

% moderncv themes style options are 'casual' (default), 'classic', 'oldstyle'
% and 'banking'
\moderncvstyle{classic}

% color options 'blue' (default), 'orange', 'green', 'red', 'purple', 'grey'
% and 'black'
\moderncvcolor{blue}

% to set the default font; use '\sfdefault' for the default sans serif font,
% '\rmdefault' for the default roman one, or any tex font name
%\renewcommand{\familydefault}{\sfdefault}         

% Uncomment for no page numbers
%\nopagenumbers{}                                  

% character encoding: if you are not using xelatex ou lualatex, replace by the
% encoding you are using
%\usepackage[utf8]{inputenc}

% adjust the page margins
\usepackage[scale=0.75]{geometry}

% if you want to change the width of the column with the dates
%\setlength{\hintscolumnwidth}{3cm}

% for the 'classic' style, if you want to force the width allocated to your
% name and avoid line breaks. be careful though, the length is normally
% calculated to avoid any overlap with your personal info; use this at your own
% typographical risks...
%\setlength{\makecvtitlenamewidth}{10cm}

% personal data
\name{Shoumik}{Palkar}
\title{Curriculum Vitae}
% For stanford
%\title{Applying to the Computer Science Dept.}
\address{239 Albany St. Apt. 2104A}{Cambridge, MA 02139}
\phone[mobile]{+1~(408)~505~6159}
\email{shoumik@mit.edu}
%\renewcommand*\httplink[2][]{{\urlstyle{sf}\expandafter\href#2}}
%\homepage{{https://www.cs.berkeley.edu/~apanda/}{www.cs.berkeley.edu/{\textasciitilde}apanda}}

% optional, remove / comment the lines if not wanted
%\social[linkedin]{john.doe}
%\social[twitter]{jdoe}
%\social[github]{jdoe}
%\extrainfo{additional information}

%'64pt' is the height the picture must be resized to, 0.4pt is the thickness of
%the frame around it (put it to 0pt for no frame) and 'picture' is the name of
%the picture file
%\photo[64pt][0.4pt]{picture}
%\quote{Some quote}

% to show numerical labels in the bibliography (default is to show no labels);
% only useful if you make citations in your resume
%\makeatletter
%\renewcommand*{\bibliographyitemlabel}{\@biblabel{\arabic{enumiv}}}

\makeatother
\renewcommand*{\bibliographyitemlabel}{[\arabic{enumiv}]}

% bibliography with mutiple entries
\usepackage{multibib}
\newcites{conference,workshop,demos,trs}{{Conferences},{Workshops},{Demos},{Technical Reports}}

%----------------------------------------------------------------------------------
%            content
%----------------------------------------------------------------------------------

\begin{document}
%-----       resume       ---------------------------------------------------------
\makecvtitle
\section{Research Interests}
Computer systems, big data, networking

\section{Education}
\cventry{2015--Present}{PhD Candidate}{Computer Science}
{Massachusetts Institute of Technology}{Cambridge, MA}{}{}

\cventry{2011--2014}{Bachelor of Science}{Computer Science and Engineering}
{University of California at Berkeley}{Berkeley, CA}{}{}

\section{Research Projects}
\subsection{E2: An Framework for Network Functions Virtualization.}
\cvitem {}
{
    \textbf{Shoumik Palkar}, Chang Lan, Sangjin Han, Keon Jang, Aurojit Panda, Sylvia Ratnasamy,
    Luigi Rizzo, and Scott Shenker.
    \begin{itemize}
        \item E2 is a system that provides a framework for NFV, responsible for dynamic scaling
        of network functions (NFs), load balancing traffic between nodes, and orchestrating
        service composition given a network policy. E2 manages a hardware cluster composed of a
        commodity switching ASIC and a rack of x86 servers.
    \end{itemize}
}

    \subsection{SoftNIC: A Software NIC to Augment Hardware}
    \cvitem{}
    {
        Sangjin Han, Keon Jang, Aurojit Panda, \textbf{Shoumik Palkar}, Dongsu Han, and Sylvia Ratnasamy.
        \begin{itemize}
            \item SoftNIC is a system built to perform complex Network Interface Card functionality in
            software with a programmable pipeline, at 40 and 100 Gbps rates, while offloading
            computation to the NIC hardware opportunistically:w
        \end{itemize}
    }

    \subsection{SDNv2}
    \cvitem{} 
    {
        Murphy McCauley et al.
        \begin{itemize}
            \item SDNv2 is a proposed internet architecture that allows ISPs to leverage commodity CPUs
            at the network edge for greater flexibility in deploying network services.
        \end{itemize}
    }

\section{Teaching Experience}
\cventry{\hspace{2em}\ \ Fall 2014}{TA for CS 168 (Undergraduate Networking)}{UC Berkeley}{Berkeley, CA}{} {
\begin{itemize}
\item Taught about 50 students in a weekly discussion section.
\item Designed problem sets, graded exams, and co-managed two projects involving
building an application-layer firewall in Python.
\end{itemize}
}

\section{Industry Experience}

\cventry{Summer 2013}{SDE Intern}{VMware Inc.}{Palo Alto, CA}{} {
\begin{itemize}
\item Developed and tested a distributed L7 Firewall in the VMware
ESX hypervisor kernel.
\end{itemize}
}

\cventry{Winter 2013}{Mobile Application Developer}{El Camino Hospital}{Mountain View, CA}{}{
\begin{itemize} 
\item Developed an iOS application for to coordinate a raffle at an event.
Integrated Amazon SimpleDB and Facebook Graph APIs.
\end{itemize}
}

\cventry{Summer 2012}{Software Engineering Intern}{Cooliris Inc.}{Palo Alto, CA}{} {
\begin{itemize} 
\item Designed and developed \emph{HashGallery}, an iOS client for Instagram,
Twitter, and Flickr image search.
\end{itemize}
}

\section{Publications}
\nociteconference{*}
\bibliographystyleconference{plainyr-rev}
\bibliographyconference{conference}

\nocitetrs{*}
\bibliographystyletrs{plainyr-rev}
\bibliographytrs{tr}

\section{References}

\end{document}
